\documentclass[12pt]{article}

\usepackage{geometry}    % required to change the page size to A4
\usepackage{graphicx}    % required for including pictures
\usepackage{float}       % allows putting an [H] in \begin{figure} to specify the exact location of the figure
\usepackage{wrapfig}     % allows in-line images such as the example fish picture
\usepackage{listings}    % source code listings
\usepackage{amsmath}     % AMS mathematical facilities for Latex
\usepackage{pstricks}    % a collection of add-ons and bug fixes for PSTricks
\usepackage{pst-circ}    % for drawing circuits
\usepackage{pst-tree}    % for drawing trees
\usepackage{pst-node}    % for drawing trees

%\usepackage{siunitx}     % a comprehensive (SI) units package
%\usepackage{tikz}
%\usepackage{circuitikz}
%\usepackage{hyperref}
%\usepackage{epsfig}
%\usepackage{fontspec}

\newcommand{\reffig}[1]{Figure~\ref{#1}}
\newcommand{\refeq}[1]{(Equation~\ref{#1})}
\newcommand{\reftab}[1]{Table~\ref{#1}}
\newcommand{\refchap}[1]{Chapter~\ref{#1}}
\newcommand{\refsec}[1]{Section~\ref{#1}}
\newcommand{\ie}{i.e.,~}
\newcommand{\eg}{e.g.,~}

\parskip 1em
\parindent 0pt

\geometry{a4paper}         % set the page size to be A4 as opposed to the default US Letter
%\usetikzlibrary{trees, snakes, shapes, arrows}
%\usetikzlibrary{calc,matrix,positioning}
%\usetikzlibrary{circuits}

%\defaultfontfeatures{Mapping=tex-text}
%\setmainfont[SmallCapsFont = Fontin SmallCaps]{Fontin}
%\graphicspath{{./figures}} % specifies the directory where pictures are stored

%----------------------------------------------------------------------------------------
% begin makebsgtitle command

\newcommand*{\makebsgtitle} {
\begingroup % create the command for including the title page in the document
  \centering % center all text
  \vspace*{\baselineskip} % white space at the top of the page

  \rule{\textwidth}{1.6pt}\vspace*{-\baselineskip}\vspace*{2pt} % thick horizontal line
  \rule{\textwidth}{0.4pt}\\[\baselineskip] % thin horizontal line

  {\LARGE\bfseries GreenDroid Configuration Network \\with Clock Domain Crossing} \\[0.2\baselineskip] % title

  \rule{\textwidth}{0.4pt}\vspace*{-\baselineskip}\vspace{3.2pt} % thin horizontal line
  \rule{\textwidth}{1.6pt}\\[\baselineskip] % thick horizontal line

  \scshape % small caps
 %Subtitle or more descriptions \\ 
  A Bespoke Silicon Group Technical Paper \\[\baselineskip] % further description

  \vspace*{5\baselineskip} % whitespace

  \includegraphics[width=0.3\textwidth]{figures/ucsd_ccores.eps}

  \vspace*{5\baselineskip} % whitespace
 %Authored by \\[\baselineskip]
 %{\Large Lu Zhang \\ Michael Taylor \par} % author list
  {\Large Bespoke Silicon Group \par} % author list
  {\scshape Department of Computer Science and Engineering \par} % year published
  {\itshape University of California,  San Diego \par} % affiliation

  \vfill % whitespace between editor names and publisher logo

 %\bsglogo \\[0.3\baselineskip] % the logo
  {\scshape February 25, 2013 - La Jolla} \\[0.3\baselineskip] % date and location
 %{\large The Publisher} \par % the publisher
\endgroup}

% end makebsgtitle command
%----------------------------------------------------------------------------------------

\begin{document}
\begin{titlepage}
\makebsgtitle % Make BSG title
\end{titlepage}

\begin{abstract}
In this technical paper we introduce the design and implementation of the
configuration network for GreenDroid chip. Rather than use a traditional scan
chain for this purpose, we develop an innovative way to broadcast configuration
data while the chip is still running and let only the target node capture data
and apply to its associated configurable pins. Since the network is likely to
operate in a different clock domain than the rest of the GreenDroid chip, we
also spend effort working on an efficient and reliable clock domain crossing
technique.
\end{abstract}
\newpage

\tableofcontents % Include a table of contents
\newpage

\section{Introduction} \label{intro}
The design of GreenDroid chip includes multiple instances of embedded IPs such
as PLLs, RAM controllers, bus interfaces and communication cores. I/O pins of
those IPs need to be connected properly, sometimes without the intervention of
the main processor. For example some PLL macro supports dynamic full PLL
configuration via its input pins, and it is better and safer to use an
independent configuration approach instead of using the main processor itself,
to prevent system from halting in case of configuration fault.

Our design of the configuration network is to safely and easily program some
embedded IP block pins of interest. Because configurations have to apply
on-the-fly to programmable pins, some of which are critical (\eg PLL pins),
this design has to be competent in the follows.
\begin{itemize}
\item Consume minimum number of chip I/O pads
\item Operate independently of the rest part of the chip
\item Apply vector configuration values atomically to pins
\item Prevent metastability in case of clock domain crossing
\end{itemize}

Therefore our design is implemented with only two I/O pads, one for its own
clock and the other for 1-bit serial input; they are fed by a configuration
driver residing outside the GreenDroid chip. The configuration network is
composed of multiple \textbf{config nodes} and \textbf{relay nodes}. Each
config node is associated to several closely located programmable pins and it
extracts configuration value from the input and apply it to those pins. A relay
node simply forwards input bit streams to output, and is used to connect
adjacent config nodes and span the configuration network. The serial input bits
are broadcast to all nodes in the network, and each config node only captures
and decodes these designated for it. A configuration value is applied to IP
block pins only when the value is completely received and decoded to ensure
atomicity. Also clock domain crossing and precautions of entering metastability
state are implemented in each config node.

The remainder of this paper is organized as follows: \refsec{design} describes
the configuration network design, including the communication protocol
(\refsec{protocol}), config node (\refsec{config-node}) and relay node
(\refsec{relay-node}). \refsec{cdc} presents our enhanced design for
clock domain crossing by using control path \textbf{synchronizer}
(\refsec{control}) and data path \textbf{stabilizer} (\refsec{data}).

\section{Design of the Configuration Network} \label{design}

\begin{figure}[h]
\centering
\scalebox{0.8} % Change this value to rescale the drawing.
{
  \pstree { \Tcircle{R} } {
    \Tdia{C}
    \pstree { \Tcircle{R} } {
      \Tdia{C}
      \pstree { \Tcircle{R} } {
        \Tdia{C}
      }
    }
    \Tdia{C}
    \pstree { \Tcircle{R} } {
      \pstree { \Tcircle{R} } {
        \pstree { \Tcircle{R} } {
          \Tdia{C}
          \pstree { \Tcircle{R} } {
            \Tdia{C}
            \Tdia{C}
          }
          \Tdia{C}
        }
      }
    }
    \Tdia{C}
  }
}

\caption{Configuration Network for GreenDroid}{R indicates a relay node and C indicates a config node}
\label{fig:config-net}
\end{figure}

\reffig{fig:config-net} illustrates an instance of the configuration network
for GreenDroid. The normal structure of this network is a tree spanned by relay
nodes, and one config node is connected as a leaf to a relay node which is
geographically close to the pins to be configured. Compared to a traditional
and linear scan chain, our design is more flexible in topology and able to
follow the system hierarchy.

The input to the root relay node of the tree is connected to the chip I/O pads.
Thereafter input data bits are sent to all branch and leaf nodes.
Multiple relay nodes could be used back-to-back to break critical path between
geographically separated config nodes. Each config node is instantiated with a
unique identifier and it only responds to the data designated for it. This
allows a particular config node to be active without interfering other nodes
even though all nodes receive the same inputs. The design therefore offers
safer configuration than a traditional scan chain does where all nodes in the
scan chain need to be configured at the same time.

\subsection{Communication Protocol} \label{protocol}
Our communication packet shown in \reftab{tab:protocol} for a config node is
comprised of a \textit{header} and \textit{framed data}. Different fields in
the header and frames in the data are separated by \textit{framing bits}, which
in our implementation is a single \textbf{0} bit.
\begin{table}[h]
\centering
  \begin{tabular}{ | c | c || c | c | c | c | c | c | }
  \hline
    \textit{f} & framed data & \textit{f} & node id & \textit{f} & packet length & \textit{f} & valid \\ \hline
  \end{tabular}
\caption{Communication Packet}{Letter \textit{f} denotes framing bits}
\label{tab:protocol}
\end{table}
During communication period the \textit{valid} field is pushed into a config
node first, followed by the remainder of header and data. In our design the
\textit{valid} field is a two-bit vector ``10'' (`0' as the least significant
bit) which indicates a valid packet has been received and that is the beginning
of a valid packet. Next to the valid field is the \textit{packet length} and
its value equals the number of bits in one complete packet (header + data,
including framing bits). The \textit{node id} is an identifier to inform the
recipient node to accept the packet. Following the header is the \textit{framed
data} composed by inserting framing bits every 8 bits in the data payload. The
length of header is the same for all config nodes, while the framed data varies
depending on the number of configurable pins connected to a particular config
node.

Because there is no routing information in the communication protocol,
configuration packets are broadcast to all config nodes in the network via the
single serial communication line. Each config node checks the valid bits and
node id in the header. If an arriving packet is valid and designated for
itself, the config node will decode the framed data and apply the value to
corresponding programmable pins; otherwise, the node simply ignores the amount
of bits specified by the packet length in header.

Bit `1' in the valid field is inserted deliberately to make an all `0' packet
not meaningful to any node; therefore we can use consecutive `0's as an NOP for
all nodes. NOPs are useful to flush the configuration network without impact on
any configured value, or to slow down the configuration data changing speed to
avoid packet loss in case the configuration network is driven by a faster clock
than the rest of the chip (refer to \refsec{control}).

\subsection{Config Node} \label{config-node}
\reffig{fig:config-node} shows the top-level diagram of a config node block.
Ports and parameters are explained in \reftab{tab:ports} and
\reftab{tab:params} respectively.

\begin{figure}[h]
\centering
\scalebox{1} % Change this value to rescale the drawing.
{
  \begin{pspicture}(0,-1.28)(5.21,1.32)
  \usefont{T1}{ppl}{m}{n}
  \rput(2.5921483,1.0080469){Config Node}
  \usefont{T1}{pcr}{m}{n}
  \rput(1.2810158,0.32304692){\small clk}
  \usefont{T1}{pcr}{m}{n}
  \rput(1.4664062,-0.2569531){\small reset}
  \usefont{T1}{pcr}{m}{n}
  \rput(1.8302637,-0.8569531){\small config\_i}
  \usefont{T1}{pcr}{m}{n}
  \rput(3.6440428,0.32304692){\small data\_o}
  \psframe[linewidth=0.04,dimen=outer](4.4,0.72)(0.8,-1.28)
  \psline[linewidth=0.02cm](0.0,0.32)(0.8,0.32)
  \psline[linewidth=0.02cm](0.0,-0.28)(0.8,-0.28)
  \psline[linewidth=0.02cm](0.0,-0.88)(0.8,-0.88)
  \psline[linewidth=0.02cm](4.4,0.32)(5.2,0.32)
  \end{pspicture}
}

\caption{Config Node}
\label{fig:config-node}
\end{figure}

\begin{table}[h]
\centering
  \begin{tabular}{ | c | c | p{9cm} | }
  \hline
    Port & Direction & Note \\ \hline
    clk & input & clock input for the destination clock domain \\ \hline
    reset & input & reset input for the destination clock domain \\ \hline
    config\_i & input & encapsulated configuration input \\ \hline
    data\_o & output & configuration value to be connected to the programmable pins
                       in the destination clock domain \\ \hline
  \end{tabular}
\caption{Config Node Ports}
\label{tab:ports}
\end{table}

The encapsulated \textit{config\_i} input is a SystemVerilog packed struct
shown below that is to ease signal assignments in Verilog coding.
\begin{lstlisting}[language=Verilog, basicstyle=\footnotesize, numbers=left, frame=single, numberstyle=\tiny]
typedef struct packed {
  logic cfg_clk;
  logic cfg_bit;
} config_s;
\end{lstlisting}

\begin{table}[h]
\centering
  \begin{tabular}{ | c | c | p{9cm} | }
  \hline
    Parameter & Type & Note \\ \hline
    id\_p & decimal & an unique identifier for each config node \\ \hline
    data\_bits\_p & decimal & number of bits in data payload
                            equal to data\_o width and the number of configurable pins \\ \hline
    default\_p & binary & safe default configuration value for pins \\ \hline
  \end{tabular}
\caption{Config Node Parameters}
\label{tab:params}
\end{table}

Apart from the parameters configurable when a config node is instantiated,
there are several local parameters same for all config nodes, which are hidden
but the user must also keep them in mind when using a config node or composing
a communication packet. Those local parameters are shown in
\reftab{tab:localparams}.
\begin{table}[h]
\centering
  \begin{tabular}{ | c | c | p{8cm} | }
  \hline
    Local Parameter & Value & Note \\ \hline
    valid bits & 2 & number of bits for the valid field in header \\ \hline
    packet length bits & 8 & number of bits to hold a packet length value \\ \hline
    node id bits & 8 & number of bits to hold a node id value \\ \hline
    data frame bits & 8 & number of bits in a data frame \\ \hline
    reset sequence bits & 10 & number of consecutive `1' bits to derive a configuration reset signal \\ \hline
    synchronizer length & 2 & number of flip-flops in the clock domain crossing synchronizer \\ \hline
  \end{tabular}
\caption{Config Node Local Parameters}
\label{tab:localparams}
\end{table}
Therefore when assigning a node id the integer value should not exceed the
maximum unsigned value that \textit{node id bits} can represent. Similarly
when composing a long packet its length value should fit in \textit{packet
length bits}.

Each config node contains a shift register used to buffer serial inputs from
config\_i and the shift register has the same structure depicted in
\reftab{tab:protocol}. Because the data payload width varies between different
config nodes, shift registers also have variable lengths to store framed
data. Every config node monitors its shift register and once it detects a valid
signal and a matching node id, it extracts payload from the framed data and
assigns corresponding bits to data\_o; if the packet is valid but the node id
does not match, the config node loads the value of
\textit{(packet length - valid bits)} into a counter and does not respond to
the shift register again (except reset, see \refsec{reset}) until the counter
decreases to zero.

Once a new configuration value is obtained it is first stored in a register of
the same clock domain as the shift register, and one cycle after the new value
is registered the config node toggles a single bit \textit{ready} signal to
inform the destination clock domain. Because the destination clock and the
configuration clock may differ, delay for one cycle here is to provide extra
time for all the data register outputs to become stable before they are read in
the destination clock domain.

By using the one-way ready signal's transition as a flag to transmit data
crossing clock domain, we benefit from two ways. One is to ensure the value
stored in the configuration clock domain registers is sampled exactly once when
we believe it's stable, and the other is to eliminate the cost of a feedback
acknowledge path, which by itself is another crossover signal and raises the
chance of circuit going metastable (refer to \refsec{meta}).

\subsection{Relay Node} \label{relay-node}
A relay node block is illustrated in \reffig{fig:relay-node}.

\begin{figure}[h]
\centering
\scalebox{1} % Change this value to rescale the drawing.
{
  \begin{pspicture}(0,-0.61780274)(5.8,0.65780276)
  \usefont{T1}{ppl}{m}{n}
  \rput(2.9354296,0.45024413){Relay Node}
  \usefont{T1}{pcr}{m}{n}
  \rput(1.9036477,-0.21475579){\small config\_i}
  \usefont{T1}{pcr}{m}{n}
  \rput(3.9498289,-0.21475579){\small config\_o}
  \psframe[linewidth=0.04,dimen=outer](5.03,0.18219727)(0.8,-0.61780274)
  \psline[linewidth=0.02cm](0.0,-0.21780273)(0.8,-0.21780273)
  \psline[linewidth=0.02cm](4.99,-0.20219727)(5.79,-0.20219727)
  \end{pspicture} 
}

\caption{Relay Node}
\label{fig:relay-node}
\end{figure}

A relay node does not perform any processing on the data packets and just
delays the inputs by one cycle and forwards them losslessly from input to
output. Relay nodes are used to construct a configuration network and break
critical path between geographically separated config nodes.

\subsection{Resets} \label{reset}
In the configuration clock domain, registers for counters and payload data need
reset at the beginning of use. However, because we only have two spare I/O pads
on the GreenDroid package, there is no way to use an additional signal from the
outside of the chip and thus the reset must be generated from the serial
inputs. To resolve this we use a sequence of bits from the shift register to
derive a reset, and those bits are defined as the least significant
\textit{reset sequence bits} mentioned in \reftab{tab:localparams}. At any
point in time, if a config node sees the reset sequence in the shift register,
a reset signal is asserted and that is used to clear the counter and load
\textit{default\_p} into the data payload register. Our reset sequence is
defined as a vector of \textit{reset sequence bits} long comprised of all `1's.
The framing bits of `0's we inserted deliberately therefore are crucial to
prevent a reset sequence from appearing in a valid configuration packet.

In the destination clock domain, a reset input is available from a port of a
config node, which is supposed to be the same reset for the rest of GreenDroid
chip. The reset for most of the GreenDroid chip is a synchronous level-sensitive
signal, while in the destination domain of a config node the reset is specially
designed to be synchronous but \textbf{edge-sensitive}. This distinction in
reset is a failsafe design in case metastability (see \refsec{meta}) seriously
impedes normal operating of GreenDroid. In this scenario we can assert the same
reset for both config nodes and the rest of the chip, and perform pin
configuration after the config node recovers and the rest circuit still remains
in reset state.

\section{Clock Domain Crossing} \label{cdc}
The configuration network is designed to work in a possibly different clock
domain than the rest of the GreenDroid chip. The clock domain crossing (CDC)
signals pose a challenging issue for design verification because traditional
functional simulation using tools like VCS is inadequate to prove crossing
signals also work properly. Within one clock domain, proper static timing
analysis (STA) can guarantee that input data changes do not violate clock setup
and hold time constraints of the sampling registers; when signals cross from
one clock domain to another unrelated clock domain, STA is not enough to rule
out metastability since CDC signals can change at any time with reference to
the destination clock edges. Therefore to avoid functional errors that can only
be detected in late design cycles, additional effort is taken to raise the
level of confidence in our design.

\subsection{Metastability} \label{meta}
The proper operation of a clocked flip-flop depends on the input signals being
stable within its required setup and hold time window. If the timing
requirements are met, the correct output signal appears after a maximum delay
of $t_{co}$ (clock-to-output time). However, if the setup or hold time constraints
are violated, the flip-flop output may take a much longer, or even infinite
time to resolve to a stable state, and the state to which it resolves to is
also non-deterministic. This unstable behavior of a flip-flop is referred to as
\textbf{Metastability} as depicted in \reffig{fig:metastability}.
Metastability events may cause wrong signals being latched in the destination
flip-flops and propagate through the rest of the circuit and therefore fail
the proper function of the chip.

\begin{figure}[ht]
\centering
\scalebox{1} % Change this value to rescale the drawing.
{
  \begin{pspicture}(0,-3.4179394)(9.230752,3.4479394)
  \psline[linewidth=0.04cm](0.39717773,1.1920606)(2.5971777,1.1920606)
  \psline[linewidth=0.04cm](2.5971777,1.1920606)(3.5971777,2.1920605)
  \psline[linewidth=0.04cm](3.5971777,2.1920605)(8.797177,2.1920605)
  \psline[linewidth=0.04cm](0.39717773,-0.20793945)(2.3971777,-0.20793945)
  \psline[linewidth=0.04cm](2.3971777,-0.20793945)(3.3971777,0.79206055)
  \psline[linewidth=0.04cm](3.3971777,0.79206055)(8.797177,0.79206055)
  \psline[linewidth=0.04cm](0.39717773,-1.6079395)(4.797178,-1.6079395)
  \psline[linewidth=0.04cm](8.797177,-0.6079395)(7.197178,-0.6079395)
  \pscustom[linewidth=0.04]
  {
  \newpath
  \moveto(4.797178,-1.6079395)
  \lineto(4.9971776,-1.6079395)
  \curveto(5.0971775,-1.6079395)(5.257178,-1.552384)(5.317178,-1.4968286)
  \curveto(5.3771777,-1.4412732)(5.4971776,-1.3301611)(5.5571775,-1.2746058)
  \curveto(5.617178,-1.2190503)(5.757178,-1.1357172)(5.8371778,-1.1079395)
  \curveto(5.9171777,-1.0801617)(6.0571775,-1.052384)(6.117178,-1.052384)
  \curveto(6.177178,-1.052384)(6.277178,-0.9690503)(6.317178,-0.88571715)
  \curveto(6.3571777,-0.802384)(6.4771776,-0.69127256)(6.5571775,-0.6634949)
  \curveto(6.637178,-0.63571715)(6.817178,-0.6079395)(6.9171777,-0.6079395)
  \curveto(7.0171776,-0.6079395)(7.137178,-0.6079395)(7.197178,-0.6079395)
  }
  \psline[linewidth=0.04cm](0.39717773,-3.0079393)(5.3971777,-3.0079393)
  \psline[linewidth=0.04cm](7.3971777,-3.0079393)(8.797177,-3.0079393)
  \pscustom[linewidth=0.04]
  {
  \newpath
  \moveto(5.3971777,-3.0079393)
  \lineto(5.5736485,-3.0079393)
  \curveto(5.6618834,-3.0079393)(5.764825,-2.8879395)(5.7795305,-2.7679396)
  \curveto(5.7942367,-2.6479394)(5.867766,-2.4979393)(5.9265895,-2.4679394)
  \curveto(5.9854126,-2.4379394)(6.117766,-2.4079394)(6.191295,-2.4079394)
  \curveto(6.264825,-2.4079394)(6.382472,-2.4679394)(6.4265895,-2.5279396)
  \curveto(6.470707,-2.5879395)(6.5736485,-2.6779394)(6.632472,-2.7079394)
  \curveto(6.691295,-2.7379394)(6.7795315,-2.7979395)(6.808943,-2.8279395)
  \curveto(6.8383546,-2.8579395)(6.8971777,-2.9179394)(6.9265895,-2.9479394)
  \curveto(6.956001,-2.9779394)(7.0295315,-3.0079393)(7.0736485,-3.0079393)
  \curveto(7.117766,-3.0079393)(7.220707,-3.0079393)(7.3971777,-3.0079393)
  }
  \psline[linewidth=0.02cm,linestyle=dashed,dash=0.16cm 0.16cm](2.5971777,3.1920605)(2.5971777,-0.40793946)
  \psline[linewidth=0.02cm,linestyle=dashed,dash=0.16cm 0.16cm](3.5971777,3.1920605)(3.5971777,-0.40793946)
  \psline[linewidth=0.02cm,linestyle=dashed,dash=0.16cm 0.16cm](2.5971777,1.1920606)(8.797177,1.1920606)
  \psline[linewidth=0.02cm,linestyle=dashed,dash=0.16cm 0.16cm](2.3971777,-0.20793945)(8.797177,-0.20793945)
  \psline[linewidth=0.02cm,linestyle=dashed,dash=0.16cm 0.16cm](5.197178,-1.6079395)(8.797177,-1.6079395)
  \psline[linewidth=0.02cm,linestyle=dashed,dash=0.16cm 0.16cm](0.39717773,2.1920605)(3.5971777,2.1920605)
  \psline[linewidth=0.02cm,linestyle=dashed,dash=0.16cm 0.16cm](0.39717773,0.79206055)(3.3971777,0.79206055)
  \psline[linewidth=0.02cm,linestyle=dashed,dash=0.16cm 0.16cm](0.39717773,-0.6079395)(6.5971775,-0.6079395)
  \psline[linewidth=0.02cm,linestyle=dashed,dash=0.16cm 0.16cm](0.39717773,-2.0079393)(8.797177,-2.0079393)
  \usefont{T1}{ppl}{m}{n}
  \rput(9.062754,2.2970605){\small 1}
  \usefont{T1}{ppl}{m}{n}
  \rput(9.083174,1.2970605){\small 0}
  \usefont{T1}{ppl}{m}{n}
  \rput(9.062754,0.8970606){\small 1}
  \usefont{T1}{ppl}{m}{n}
  \rput(9.083174,-0.10293946){\small 0}
  \usefont{T1}{ppl}{m}{n}
  \rput(9.062754,-0.50293946){\small 1}
  \usefont{T1}{ppl}{m}{n}
  \rput(9.083174,-1.5029395){\small 0}
  \usefont{T1}{ppl}{m}{n}
  \rput(9.062754,-1.9029394){\small 1}
  \usefont{T1}{ppl}{m}{n}
  \rput(9.083174,-2.9029396){\small 0}
  \usefont{T1}{ppl}{m}{n}
  \rput(0.44458497,1.4970605){\small Clock}
  \usefont{T1}{ppl}{m}{n}
  \rput(0.98270994,0.09706055){\small Async Input}
  \usefont{T1}{ppl}{m}{n}
  \rput(1.7363964,-1.3029394){\small Metastable Output (a)}
  \usefont{T1}{ppl}{m}{n}
  \rput(1.7563965,-2.7029395){\small Metastable Output (b)}
  \psline[linewidth=0.02cm,arrowsize=0.05291667cm 2.0,arrowlength=1.4,arrowinset=0.4]{<->}(2.5971777,2.9920607)(3.1971776,2.9920607)
  \psline[linewidth=0.02cm,arrowsize=0.05291667cm 2.0,arrowlength=1.4,arrowinset=0.4]{<->}(3.1971776,2.9920607)(3.5971777,2.9920607)
  \psline[linewidth=0.02cm,linestyle=dashed,dash=0.16cm 0.16cm](3.1971776,3.1920605)(3.1971776,-3.4079394)
  \psline[linewidth=0.02cm,linestyle=dashed,dash=0.16cm 0.16cm](5.9971776,3.1920605)(5.9971776,-3.4079394)
  \psline[linewidth=0.02cm,arrowsize=0.05291667cm 2.0,arrowlength=1.4,arrowinset=0.4]{<->}(3.1971776,-0.80793947)(5.9971776,-0.80793947)
  \usefont{T1}{ppl}{m}{n}
  \rput(2.8858105,3.2770605){\scriptsize $t_{su}$}
  \usefont{T1}{ppl}{m}{n}
  \rput(3.416631,3.2770605){\scriptsize $t_h$}
  \usefont{T1}{ppl}{m}{n}
  \rput(4.590869,-1.0229395){\scriptsize $t_{co}$}
  \end{pspicture}
}


\caption{The Metastability Behavior}
\label{fig:metastability}
\end{figure}

Designers often use the following synchronization techniques
\cite{cadence2004tp} to reduce the chance of metastability.
\begin{itemize}
\item A chain of flip-flops for 1-bit crossover signal
\item Gray-encoding for vector crossover signals
\item Handshaking synchronizers for control signals and MUXes for data
\item Dual-clock FIFO
\end{itemize}

Variations of these common techniques are used in different design scenarios.
In our work, we use a combination of a flip-flip sequence and a data MUX, on
top of that we develop a customized synchronizing protocol for the control path
and clock domain crossing stabilizers along the data path. The RTL diagram of
our design is depicted in \reffig{fig:our-cdc}.

\begin{figure}
\centering
\scalebox{0.8} % Change this value to rescale the drawing.
{
  \begin{pspicture}(0,-3.01)(16.514824,3.02)
  \psframe[linewidth=0.04,dimen=outer](2.0848243,2.2)(0.8848242,0.6)
  \usefont{T1}{pcr}{m}{n}
  \rput(1.1665039,1.905){\small D}
  \usefont{T1}{pcr}{m}{n}
  \rput(1.7698243,1.905){\small Q}
  \psline[linewidth=0.02](0.92482424,1.54)(1.2048242,1.42)(0.92482424,1.28)(0.92482424,1.28)
  \psframe[linewidth=0.04,dimen=outer](2.0848243,-0.2)(0.8848242,-1.8)
  \usefont{T1}{pcr}{m}{n}
  \rput(1.1665039,-0.495){\small D}
  \usefont{T1}{pcr}{m}{n}
  \rput(1.7698243,-0.495){\small Q}
  \psline[linewidth=0.02](0.92482424,-0.86)(1.2048242,-0.98)(0.92482424,-1.12)(0.92482424,-1.12)
  \psframe[linewidth=0.04,dimen=outer](4.084824,-0.2)(2.8848243,-1.8)
  \usefont{T1}{pcr}{m}{n}
  \rput(3.166504,-0.495){\small D}
  \usefont{T1}{pcr}{m}{n}
  \rput(3.7698243,-0.495){\small Q}
  \psline[linewidth=0.02](2.9248242,-0.86)(3.2048242,-0.98)(2.9248242,-1.12)(2.9248242,-1.12)
  \psline[linewidth=0.02cm,linestyle=dashed,dash=0.16cm 0.16cm](5.2848244,2.6)(5.2848244,-3.0)
  \psframe[linewidth=0.04,dimen=outer](7.684824,-0.2)(6.484824,-1.8)
  \usefont{T1}{pcr}{m}{n}
  \rput(6.766504,-0.495){\small D}
  \usefont{T1}{pcr}{m}{n}
  \rput(7.3698244,-0.495){\small Q}
  \psline[linewidth=0.02](6.524824,-0.86)(6.8048244,-0.98)(6.524824,-1.12)(6.524824,-1.12)
  \psframe[linewidth=0.04,dimen=outer](10.284824,-0.2)(9.084825,-1.8)
  \usefont{T1}{pcr}{m}{n}
  \rput(9.366504,-0.495){\small D}
  \usefont{T1}{pcr}{m}{n}
  \rput(9.969824,-0.495){\small Q}
  \psline[linewidth=0.02](9.124825,-0.86)(9.404824,-0.98)(9.124825,-1.12)(9.124825,-1.12)
  \psframe[linewidth=0.04,dimen=outer](12.284824,-0.2)(11.084825,-1.8)
  \usefont{T1}{pcr}{m}{n}
  \rput(11.366504,-0.495){\small D}
  \usefont{T1}{pcr}{m}{n}
  \rput(11.969824,-0.495){\small Q}
  \psline[linewidth=0.02](11.124825,-0.86)(11.404824,-0.98)(11.124825,-1.12)(11.124825,-1.12)
  \psframe[linewidth=0.04,dimen=outer](14.284824,-0.2)(13.084825,-1.8)
  \usefont{T1}{pcr}{m}{n}
  \rput(13.366504,-0.495){\small D}
  \usefont{T1}{pcr}{m}{n}
  \rput(13.969824,-0.495){\small Q}
  \psline[linewidth=0.02](13.124825,-0.86)(13.404824,-0.98)(13.124825,-1.12)(13.124825,-1.12)
  \psline[linewidth=0.02cm,arrowsize=0.05291667cm 2.0,arrowlength=1.4,arrowinset=0.4]{->}(0.08482422,-0.6)(0.8848242,-0.6)
  \psline[linewidth=0.02cm,arrowsize=0.05291667cm 2.0,arrowlength=1.4,arrowinset=0.4]{->}(2.0848243,-0.6)(2.8848243,-0.6)
  \psline[linewidth=0.02cm,arrowsize=0.05291667cm 2.0,arrowlength=1.4,arrowinset=0.4]{->}(4.084824,-0.6)(6.484824,-0.6)
  \psline[linewidth=0.02cm,arrowsize=0.05291667cm 2.0,arrowlength=1.4,arrowinset=0.4]{->}(10.284824,-0.6)(11.084825,-0.6)
  \psline[linewidth=0.02cm,arrowsize=0.05291667cm 2.0,arrowlength=1.4,arrowinset=0.4]{->}(12.284824,-0.6)(13.084825,-0.6)
  \psframe[linewidth=0.04,dimen=outer](12.284824,2.2)(11.084825,0.6)
  \usefont{T1}{pcr}{m}{n}
  \rput(11.366504,1.905){\small D}
  \usefont{T1}{pcr}{m}{n}
  \rput(11.969824,1.905){\small Q}
  \psline[linewidth=0.02](11.124825,1.54)(11.404824,1.42)(11.124825,1.28)(11.124825,1.28)
  \psframe[linewidth=0.04,dimen=outer](7.8848243,2.2)(6.2848244,1.4)
  \usefont{T1}{ptm}{m}{n}
  \rput(7.069121,1.8){\footnotesize Stablizer}
  \psline[linewidth=0.04](9.284824,0.8)(9.684824,1.12)(9.684824,2.08)(9.284824,2.4)(9.284824,0.8)
  \psframe[linewidth=0.04,dimen=outer](16.084824,-0.4)(15.084825,-1.2)
  \usefont{T1}{ptm}{m}{n}
  \rput(15.578408,-0.815){\scriptsize XOR}
  \psline[linewidth=0.02cm,arrowsize=0.05291667cm 2.0,arrowlength=1.4,arrowinset=0.4]{->}(14.284824,-0.6)(15.084825,-0.6)
  \psline[linewidth=0.02,arrowsize=0.05291667cm 2.0,arrowlength=1.4,arrowinset=0.4]{->}(12.684824,-0.6)(12.684824,-2.2)(14.684824,-2.2)(14.684824,-1.0)(15.084825,-1.0)
  \psdots[dotsize=0.12](12.684824,-0.6)
  \psline[linewidth=0.02,arrowsize=0.05291667cm 2.0,arrowlength=1.4,arrowinset=0.4]{->}(16.084824,-0.8)(16.484825,-0.8)(16.484825,0.2)(9.484824,0.2)(9.484824,1.0)
  \usefont{T1}{pcr}{m}{n}
  \rput(9.408906,1.205){\small 1}
  \usefont{T1}{pcr}{m}{n}
  \rput(9.408823,1.985){\small 0}
  \psline[linewidth=0.06cm,arrowsize=0.05291667cm 2.0,arrowlength=1.4,arrowinset=0.4]{->}(0.08482422,1.8)(0.8848242,1.8)
  \psline[linewidth=0.06cm,arrowsize=0.05291667cm 2.0,arrowlength=1.4,arrowinset=0.4]{->}(2.0848243,1.8)(6.2848244,1.8)
  \psline[linewidth=0.06,arrowsize=0.05291667cm 2.0,arrowlength=1.4,arrowinset=0.4]{->}(7.8848243,1.8)(8.484824,1.8)(8.484824,1.2)(9.284824,1.2)
  \psline[linewidth=0.06,arrowsize=0.05291667cm 2.0,arrowlength=1.4,arrowinset=0.4]{->}(12.284824,1.8)(12.684824,1.8)(12.684824,2.8)(8.884824,2.8)(8.884824,2.0)(9.284824,2.0)
  \psline[linewidth=0.06,arrowsize=0.05291667cm 2.0,arrowlength=1.4,arrowinset=0.4]{->}(9.684824,1.6)(10.284824,1.6)(10.284824,1.8)(11.084825,1.8)
  \psline[linewidth=0.04cm](0.4848242,2.0)(0.28482422,1.6)
  \psline[linewidth=0.04cm](4.2848244,2.0)(4.084824,1.6)
  \psline[linewidth=0.04cm](8.284824,2.0)(8.084825,1.6)
  \psline[linewidth=0.04cm](10.084825,1.8)(9.884824,1.4)
  \psline[linewidth=0.04cm](11.084825,2.8)(10.884824,2.8)
  \psline[linewidth=0.04cm](10.684824,2.6)(10.884824,3.0)
  \usefont{T1}{ppl}{m}{n}
  \rput(0.7411328,2.445){\scriptsize config data}
  \usefont{T1}{ppl}{m}{n}
  \rput(0.39368165,0.005){\scriptsize ready}
  \psline[linewidth=0.02cm,arrowsize=0.05291667cm 2.0,arrowlength=1.4,arrowinset=0.4]{->}(7.684824,-0.6)(8.084825,-0.6)
  \psline[linewidth=0.02cm,arrowsize=0.05291667cm 2.0,arrowlength=1.4,arrowinset=0.4]{->}(8.684824,-0.6)(9.084825,-0.6)
  \psline[linewidth=0.02cm,linestyle=dashed,dash=0.16cm 0.16cm](8.084825,-0.6)(8.884824,-0.6)
  \psframe[linewidth=0.02,linestyle=dashed,dash=0.16cm 0.16cm,dimen=outer](10.684824,0.0)(6.084824,-2.4)
  \usefont{T1}{ptm}{m}{n}
  \rput(8.559121,-2.2){\footnotesize Synchronizer}
  \psline[linewidth=0.06cm,arrowsize=0.05291667cm 2.0,arrowlength=1.4,arrowinset=0.4]{->}(12.684824,1.8)(16.484825,1.8)
  \psdots[dotsize=0.12](12.684824,1.8)
  \usefont{T1}{ptm}{m}{n}
  \rput(15.179902,2.2){\footnotesize to configure pins}
  \usefont{T1}{ptm}{m}{n}
  \rput(3.0286524,-2.8){\footnotesize configuration clock domain}
  \usefont{T1}{ptm}{m}{n}
  \rput(7.2688084,-2.8){\footnotesize destination clock domain}
  \psline[linewidth=0.02,arrowsize=0.05291667cm 2.0,arrowlength=1.4,arrowinset=0.4]{->}(9.484824,0.2)(7.084824,0.2)(7.084824,1.4)
  \psdots[dotsize=0.12](9.484824,0.2)
  \end{pspicture}
}


\caption{Our Clock-Domain Crossing Circuit}
\label{fig:our-cdc}
\end{figure}

\subsection{Control Path Synchronizer} \label{control}
As described in \refsec{design}, the source clock domain uses a single bit
alternate \textit{ready} signal to indicate that a new configuration value is
ready and stable to be sampled. In the destination clock domain, two flip-flops
are XOR-ed to detect the edge of a transitioning ready signal. A
parameterizable-length (at least two) sequence of flip-flops is inserted
between the source ready and the first edge-detecting flip-flop to provide
extra amount of time for the possible metastable value to resolve and lowers
the probability that the design fails. %\cite{veendrick1980jssc}

Successful operation of our design is based on the assumption that data in the
source clock domain registers does not change before it gets sampled in the
destination domain. Since there is no feedback loop in our design, we guarantee
this by feeding the configuration network with properly designed input sequence
to ensure data is stable during the communication period, which is the length
of the synchronizer chain plus three flip-flops (one delay flip-flop in the
source clock domain and the two edge-detecting flip-flops in the destination).

Compared to a full handshaking communication, the one-way control signal design
has the following advantages:
\begin{itemize}
\item Save logic to implement a feedback path.
\item Reduce the chance of circuit metastability because the feedback path also
      crosses clock domains.
\item Respond faster and deliver higher throughputs in transferring data.
\end{itemize}
and the only disadvantage is that it cannot stall the source data pipeline and
may lose packets if data from source clock domain is changing too frequent.
In our design, we are able to lower the configuration data changing frequency
by inserting NOPs from the input of the network; and therefore eliminate the
chance of data loss, while taking advantage of the benefits from the one-way
crossing.

Although the use of synchronizers can not completely eliminate the chance of
entering metastable state, the probabilistic function of Mean Time Between
Failure (MTBF, refer to \refsec{mtbf}) qualifies the circuit to be continuously
operating in acceptable duration of time. In RTL development stage of a chip,
users of this design can increase the MTBF by extending the length of the
synchronizer chain.

\subsection{Data Path Multiplexer} \label{data}
The control signal is typically considered to be safe by using the
aforementioned synchronizers, and our design makes sure the asynchronous data
path is through when a transition of ready is detected, \ie when the
\textit{select} input to the data MUX is logic 1. Nevertheless, one commonly
overlooked potential cause of metastability is the use of this asynchronous
data MUX when its select input is logic 0. Ideally, the MUX should shut off the
asynchronous data path completely when its select input is logic 0, but since
the physical implementations of MUXes vary between different process design
kits, we are not convinced that the MUX always works as we expected.  During
normal operation of the circuit, toggling of ready does not happen
frequently and this means the destination register at the crossing boundary
spends most of its time pulling data from the MUX's data path selected by logic
0. Although this data path is synchronous with respect to the destination
clock, we need extra measures to make sure the stability of this data path is
not interfered by the asynchronous data input. Interferences may also cause
unstable signals reach the register's input within its setup-hold time window.
Therefore we introduce the \textit{clock domain crossing stabilizers} right
before the MUX's asynchronous data input to provide extra protection for the
synchronous logic.

\begin{figure}[h]
\centering
\scalebox{1} % Change this value to rescale the drawing.
{
  \begin{pspicture}(0,-2.8552785)(3.7996485,2.9)
  \psline[linewidth=0.04cm](1.4841797,2.24)(1.8041797,2.24)
  \psline[linewidth=0.04](1.6441798,2.24)(1.6441798,1.76)(1.3241796,1.76)(1.3241796,1.12)(1.6441798,1.12)(1.6441798,0.64)(2.2841797,0.64)(2.2841797,-0.32)(1.9641796,-0.32)(1.9641796,-0.96)(2.2841797,-0.96)(2.2841797,-1.44)(1.9641796,-1.44)(1.9641796,-2.08)(2.2841797,-2.08)(2.2841797,-2.56)
  \psline[linewidth=0.04](2.2841797,0.64)(2.9241798,0.64)(2.9241798,1.12)(2.6041799,1.12)(2.6041799,1.76)(2.9241798,1.76)(2.9241798,2.24)
  \psline[linewidth=0.04cm](2.7641797,2.24)(3.0841799,2.24)
  \psline[linewidth=0.04cm](1.1641797,1.76)(1.1641797,1.12)
  \psline[linewidth=0.04cm](2.4441798,1.76)(2.4441798,1.12)
  \psline[linewidth=0.04cm](1.8041797,-0.32)(1.8041797,-0.96)
  \psline[linewidth=0.04cm](1.8041797,-1.44)(1.8041797,-2.08)
  \rput{-180.0}(4.56836,-5.44){\pstriangle[linewidth=0.04,dimen=outer](2.28418,-2.88)(0.32,0.32)}
  \usefont{T1}{pcr}{m}{n}
  \rput(0.07488281,-1.495){B}
  \usefont{T1}{pcr}{m}{n}
  \rput(0.13535157,1.705){A}
  \psline[linewidth=0.04](0.84417975,1.44)(0.84417975,-0.64)(1.8041797,-0.64)
  \psline[linewidth=0.04cm,dotsize=0.07055555cm 2.0]{-oo}(2.1241798,1.44)(2.4441798,1.44)
  \psline[linewidth=0.04cm,dotsize=0.07055555cm 2.0]{-oo}(0.044179693,1.44)(1.1641797,1.44)
  \psline[linewidth=0.04](2.1241798,1.44)(2.1241798,2.88)(0.5241797,2.88)(0.5241797,-1.76)(1.8041797,-1.76)
  \psline[linewidth=0.04cm](0.5241797,-1.76)(0.044179693,-1.76)
  \psdots[dotsize=0.12](0.5241797,-1.76)
  \psdots[dotsize=0.12](0.84417975,1.44)
  \usefont{T1}{pcr}{m}{n}
  \rput(2.7630959,2.485){\scriptsize $V_{dd}$}
  \usefont{T1}{pcr}{m}{n}
  \rput(1.4830958,2.485){\scriptsize $V_{dd}$}
  \usefont{T1}{pcr}{m}{n}
  \rput(2.290546,-0.6878353){\scriptsize N1}
  \usefont{T1}{pcr}{m}{n}
  \rput(2.290546,-1.7596855){\scriptsize N2}
  \usefont{T1}{pcr}{m}{n}
  \rput(1.7067704,1.4614192){\scriptsize P1}
  \usefont{T1}{pcr}{m}{n}
  \rput(2.9615715,1.4614192){\scriptsize P2}
  \psdots[dotsize=0.12](2.2996485,0.62)
  \psline[linewidth=0.04cm,arrowsize=0.05291667cm 2.0,arrowlength=1.4,arrowinset=0.4]{->}(2.2996485,0.1)(3.7796485,0.1)
  \usefont{T1}{pcr}{m}{n}
  \rput(3.220244,-0.075){\scriptsize Output}
  \psdots[dotsize=0.12](2.2798243,0.1)
  \end{pspicture}
}

\caption{6-MOSFET NAND2 Gate}
\label{fig:nand2}
\end{figure}
Our stabilizer is implemented by exactly one NAND2 gate, whose behavior is
predictable, and implementation is simple enough to be done in typical six
MOSFETs, as shown in \reffig{fig:nand2}. When one input, for example B, to this
NAND2 gate is stable logic 0, obviously N-MOS N2 is shut off and no matter how
input A varies, the NAND2 output is strongly charged by $V_{dd}$ and immune to
changes of input A. By contrast, if B is stable logic 1, then N-MOS N2 is on,
and now if input A voltage varies, the NAND2 output also changes accordingly
until it resolves to a state determined by input A. Corresponding voltage
divider models are presented in \reffig{fig:nand2-in0} and
\reffig{fig:nand2-in1}, respectively. Same analysis also works if we choose A
as the stable input.

\begin{figure}[ht]
\centering
\scalebox{1} % Change this value to rescale the drawing.
{
  \begin{pspicture}(0,-2.7083156)(6.9017186,2.7730372)
  \psline[linewidth=0.04cm](1.4417187,2.3869631)(1.7617188,2.3869631)
  \psline[linewidth=0.04](1.6017188,2.3869631)(1.6017188,1.906963)(1.2817187,1.906963)(1.2817187,1.266963)(1.6017188,1.266963)(1.6017188,0.786963)(2.2417188,0.786963)(2.2417188,-0.17303696)(1.9217187,-0.17303696)(1.9217187,-0.813037)(2.2417188,-0.813037)(2.2417188,-1.133037)
  \psline[linewidth=0.02cm](2.0817187,-1.2930368)(2.4017189,-1.773037)
  \psline[linewidth=0.02cm](2.4017189,-1.2930368)(2.0817187,-1.773037)
  \psline[linewidth=0.04cm](2.2417188,-1.933037)(2.2417188,-2.4130368)
  \rput{-180.0}(4.4834375,-5.1460743){\pstriangle[linewidth=0.04,dimen=outer](2.2417188,-2.733037)(0.32,0.32)}
  \psline[linewidth=0.04cm](1.1217188,1.906963)(1.1217188,1.266963)
  \psline[linewidth=0.04cm](1.7617188,-0.17303696)(1.7617188,-0.813037)
  \psline[linewidth=0.04cm,dotsize=0.07055555cm 2.0]{-oo}(0.0017187487,1.586963)(1.1217188,1.586963)
  \psline[linewidth=0.04](0.8017188,1.586963)(0.8017188,-0.49303696)(1.7617188,-0.49303696)
  \psline[linewidth=0.02](2.2417188,0.786963)(2.8817186,0.786963)(2.8817186,2.3869631)
  \psline[linewidth=0.04cm](2.7217188,2.3869631)(3.0417187,2.3869631)
  \psdots[dotsize=0.12](2.2417188,0.786963)
  \psdots[dotsize=0.12](0.8017188,1.586963)
  \usefont{T1}{pcr}{m}{n}
  \rput(0.10230469,1.851963){A}
  \usefont{T1}{pcr}{m}{n}
  \rput(1.5472656,2.631963){\scriptsize $V_{dd}$}
  \usefont{T1}{pcr}{m}{n}
  \rput(2.8272655,2.631963){\scriptsize $V_{dd}$}
  \psline[linewidth=0.04cm,arrowsize=0.05291667cm 2.0,arrowlength=1.4,arrowinset=0.4]{->}(2.2417188,0.30696303)(3.6817188,0.30696303)
  \psdots[dotsize=0.12](2.2417188,0.30696303)
  \usefont{T1}{pcr}{m}{n}
  \rput(3.1688867,0.13196304){\scriptsize Output}
  \usefont{T1}{pcr}{m}{n}
  \rput(2.2731423,-0.5049328){\scriptsize N1}
  \usefont{T1}{pcr}{m}{n}
  \rput(2.690374,-1.5248139){\scriptsize N2}
  \usefont{T1}{pcr}{m}{n}
  \rput(1.6173949,1.5641589){\scriptsize P1}
  \usefont{T1}{pcr}{m}{n}
  \rput(3.196233,1.5641589){\scriptsize P2}
  \psline[linewidth=0.04cm](5.2817187,2.3869631)(5.601719,2.3869631)
  \psline[linewidth=0.04,arrowsize=0.05291667cm 2.0,arrowlength=1.4,arrowinset=0.4]{->}(5.4417186,2.3869631)(5.4417186,0.786963)(5.4417186,0.30696303)(6.8817186,0.30696303)
  \usefont{T1}{pcr}{m}{n}
  \rput(5.3872657,2.631963){\scriptsize $V_{dd}$}
  \usefont{T1}{pcr}{m}{n}
  \rput(6.368887,0.13196304){\scriptsize Output}
  \end{pspicture}
}

\caption{NAND2 Gate with a Stable Logic 0 Input}
\label{fig:nand2-in0}
\end{figure}

\begin{figure}[ht]
\centering
\scalebox{1} % Change this value to rescale the drawing.
{
  \begin{pspicture}(0,-2.7083156)(7.541719,2.7730372)
  \psline[linewidth=0.04cm](1.4417187,2.3869631)(1.7617188,2.3869631)
  \psline[linewidth=0.04](1.6017188,2.3869631)(1.6017188,1.906963)(1.2817187,1.906963)(1.2817187,1.266963)(1.6017188,1.266963)(1.6017188,0.786963)(2.2417188,0.786963)(2.2417188,-0.17303696)(1.9217187,-0.17303696)(1.9217187,-0.813037)(2.2417188,-0.813037)(2.2417188,-1.133037)
  \psline[linewidth=0.04cm](2.2417188,-1.933037)(2.2417188,-2.4130368)
  \rput{-180.0}(4.4834375,-5.1460743){\pstriangle[linewidth=0.04,dimen=outer](2.2417188,-2.733037)(0.32,0.32)}
  \psline[linewidth=0.04cm](1.1217188,1.906963)(1.1217188,1.266963)
  \psline[linewidth=0.04cm](1.7617188,-0.17303696)(1.7617188,-0.813037)
  \psline[linewidth=0.04cm,dotsize=0.07055555cm 2.0]{-oo}(0.0017187487,1.586963)(1.1217188,1.586963)
  \psline[linewidth=0.04](0.8017188,1.586963)(0.8017188,-0.49303696)(1.7617188,-0.49303696)
  \psline[linewidth=0.04cm](2.7217188,2.3869631)(3.0417187,2.3869631)
  \psdots[dotsize=0.12](2.2417188,0.786963)
  \psdots[dotsize=0.12](0.8017188,1.586963)
  \usefont{T1}{pcr}{m}{n}
  \rput(0.10230469,1.851963){A}
  \usefont{T1}{pcr}{m}{n}
  \rput(1.5472656,2.631963){\scriptsize $V_{dd}$}
  \usefont{T1}{pcr}{m}{n}
  \rput(2.8272655,2.631963){\scriptsize $V_{dd}$}
  \psline[linewidth=0.04cm,arrowsize=0.05291667cm 2.0,arrowlength=1.4,arrowinset=0.4]{->}(2.2417188,0.30696303)(3.6817188,0.30696303)
  \psdots[dotsize=0.12](2.2417188,0.30696303)
  \usefont{T1}{pcr}{m}{n}
  \rput(3.1688867,0.13196304){\scriptsize Output}
  \usefont{T1}{pcr}{m}{n}
  \rput(2.267437,-0.46986395){\scriptsize N1}
  \usefont{T1}{pcr}{m}{n}
  \rput(2.607385,-1.5803776){\scriptsize N2}
  \usefont{T1}{pcr}{m}{n}
  \rput(1.6047277,1.6133491){\scriptsize P1}
  \usefont{T1}{pcr}{m}{n}
  \rput(3.196163,1.6133491){\scriptsize P2}
  \psline[linewidth=0.04cm](5.2817187,2.3869631)(5.601719,2.3869631)
  \usefont{T1}{pcr}{m}{n}
  \rput(5.3872657,2.631963){\scriptsize $V_{dd}$}
  \usefont{T1}{pcr}{m}{n}
  \rput(7.008887,0.13196304){\scriptsize Output}
  \psline[linewidth=0.02cm](2.2417188,-1.133037)(2.2417188,-1.933037)
  \psline[linewidth=0.02cm](2.8817186,2.3869631)(2.8817186,2.0669632)
  \psline[linewidth=0.02cm](3.0417187,1.906963)(2.7217188,1.4269631)
  \psline[linewidth=0.02cm](2.7217188,1.906963)(3.0417187,1.4269631)
  \psline[linewidth=0.02](2.8817186,1.266963)(2.8817186,0.786963)(2.2417188,0.786963)
  \psline[linewidth=0.04cm](5.4417186,2.3869631)(5.4417186,1.906963)
  \psline[linewidth=0.02](5.464576,1.906963)(5.601719,1.8269632)(5.2817187,1.746963)(5.601719,1.666963)(5.2817187,1.586963)(5.601719,1.506963)(5.2817187,1.4269631)(5.601719,1.346963)(5.4417186,1.266963)
  \psline[linewidth=0.04](5.4417186,1.266963)(5.4417186,0.786963)(6.081719,0.786963)(6.081719,0.30696303)(6.081719,-0.17303696)
  \psline[linewidth=0.02](6.1045756,-0.17303696)(6.241719,-0.25303695)(5.9217186,-0.33303696)(6.241719,-0.41303697)(5.9217186,-0.49303696)(6.241719,-0.57303697)(5.9217186,-0.653037)(6.241719,-0.73303694)(6.081719,-0.813037)
  \psline[linewidth=0.02cm,arrowsize=0.05291667cm 2.0,arrowlength=1.4,arrowinset=0.4]{->}(5.2817187,1.266963)(5.7617188,1.906963)
  \psline[linewidth=0.02cm,arrowsize=0.05291667cm 2.0,arrowlength=1.4,arrowinset=0.4]{->}(5.9217186,-0.813037)(6.4017186,-0.17303696)
  \psline[linewidth=0.04cm](6.081719,-0.813037)(6.081719,-1.773037)
  \rput{-180.0}(12.163438,-3.866074){\pstriangle[linewidth=0.04,dimen=outer](6.081719,-2.0930371)(0.32,0.32)}
  \psline[linewidth=0.04cm,arrowsize=0.05291667cm 2.0,arrowlength=1.4,arrowinset=0.4]{->}(6.081719,0.30696303)(7.521719,0.30696303)
  \psdots[dotsize=0.12](6.081719,0.30696303)
  \end{pspicture}
}

\caption{NAND2 Gate with a Stable Logic 1 Input}
\label{fig:nand2-in1}
\end{figure}

Deliberately we connect one input of the stabilizer to the same select signal
of the data MUX, and the other input to one asynchronous data line. By this
means, we believe the asynchronous data fluctuations are well filtered by the
stabilizers, and passed to the destination registers only when the crossover
data is stable, which is signaled by a logic 1 on the select line.

Noticeably valid configuration signals are inverted after NAND2 gates, and
recovery inverters are inserted after the sampling registers. At the cost of a
few more MOSFETs, we raise our confidence in this clock domain crossing design.
Use of such stabilizers is \textbf{not} merely bound to our circuit and can
also be applied to other clock domain crossing cases whenever crossover data
needs to be blocked.

\subsection{Mean Time Between Failure} \label{mtbf}
Although the use of synchronizers can reduce the chance that
metastable signals cause the circuit fail, the chance cannot be completely
avoided in a clock domain crossing design. Typically the metric of MTBF
is employed to evaluate the likelihood of a circuit fail.

Entering a metastable state is a probabilistic function with reference to the
destination clock frequency, the asynchronous signal transitioning frequency
and constants that defines the window in which a transition can cause
metastability. A typical formula \cite{altera2009wp} \cite{cadence2004tp}
\cite{chen2010fpga} \cite{dike1999miller} to calculate the MTBF of a
two-flip-flop synchronizer sampling asynchronous data is given in
\refeq{eq:mtbf-2ff}.

\begin{equation} \label{eq:mtbf-2ff}
MTBF = \frac{e^{\frac{T_r}{C_2}}} {C_1 \times f_{dst} \times f_{data}}
\end{equation}

In \refeq{eq:mtbf-2ff}, $C_1$ and $C_2$ are constants that depend on the
tape-out process and the operating conditions of the flip-flops; $T_r$ is
the maximum allowed amount of time for the first flip-flop to resolve a metastable
state; and $f_{dst}$ and $f_{data}$ are the frequencies of the destination
clock and the asynchronous data input, respectively \cite{cadence2004tp}. If the
actual resolution time of the first flip-flip exceeds $T_r$, the
metastable state will propagate to the second one and probably cause functional
failures of the downstream circuit. The $T_r$ for an N-flip-flop
synchronizer is the sum of $T_r$ for each flip-flop in this chain, or that
expressed in \refeq{eq:mtbf-nff}.

\begin{equation} \label{eq:mtbf-nff}
MTBF = \frac{e^{\frac{\sum_{0}^{N - 1} T_{r,i}}{C_2}}} {C_1 \times f_{dst} \times f_{data}}
\end{equation}

Constants $C_1$ (sometimes referred to as $\tau$) and $C_2$ (sometimes referred
to as $T_0$ or $T_w$) need to be obtained by contacting the chip foundry, or
extracted through experiments \cite{baghini2002impact} \cite{chen2010fpga}.
Apparently one straightforward way to improve the MTBF is to extend the number
of stages in the synchronizer chain, because the accumulated $T_{r,i}$ will
lead to greater MTBF value. Designers can also use special metastable hardened
flops for increasing the MTBF \cite{cadence2004tp}, by varying the $C_1$
and/or $C_2$ constants. Once the design is taped out, we can also manipulate
$f_{dst}$ or more effectively $f_{data}$ to improve MTBF.

\section{Conclusions and Future Work} \label{conclusions}
This technical paper has presented the configuration network design for our
GreenDroid chip. We believe this design is better and safer than a traditional
scan chain for the same purpose. We have also taken strong measures in our
design trying to reduce the chance of metastability caused by clock domain
crossing. The design will be evaluated by FPGA simulation and eventually
validated by testing the GreenDroid chip. At this point in time, this
configuration network is write-only, which means data can only be pushed into
configurable targets. Adding support for reading output status of IPs is also
useful for system status monitoring.

\bibliographystyle{abbrv}
\bibliography{20130219}

\end{document}
